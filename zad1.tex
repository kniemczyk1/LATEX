\documentclass[12pt,a4paper]{article}
\usepackage[MeX]{polski} 
\usepackage[utf8]{inputenc}  
\usepackage{graphicx}
\graphicspath{ {./images/} }
\usepackage{hyperref}
\usepackage{float}
\title{TYTUŁ PRACY}
\author{KAMIL NIEMCZYK}
\begin{document}
	\maketitle
	\newpage
	\begin{abstract}
		Obecnie forma papieskiego samochodu zbliżona jest do zwykłego samochodu typu pick-up z tylną przeszkloną częścią w kształcie sześcianu. Konstrukcję auta specjalnie wzmocniono (nadwozie, silnik), a część tylną, w której podczas przejazdu przebywa papież, dodatkowo opancerzono (od czasu zamachu na papieża Jana Pawła II z dnia 13 maja 1981). Jednak papież Franciszek zrezygnował z opancerzeń.
	\end{abstract}
	\newpage
	\tableofcontents
	\newpage
	\part *{Rozdział 1}
	\section{Zaczecie}
	Lorem Ipsum is simply dummy text of the printing and typesetting industry. Lorem Ipsum has been the industry's standard dummy text ever since the 1500s, \underline{when an unknown printer} took a galley of type and scrambled it to make a type specimen book. It has survived not only five centuries, but also the leap into electronic typesetting, remaining essentially unchanged. 
	Contrary to popular belief, Lorem Ipsum is not simply random text. It has roots in a piece of classical Latin literature from 45 BC, making it over 2000 years old. Richard McClintock, a Latin professor at Hampden-Sydney College in Virginia, looked up one of the more obscure Latin words, consectetur, from a Lorem Ipsum passage, and going through the cites of the word in classical literature, discovered the undoubtable source. Lorem Ipsum comes from sections 1.10.32 and 1.10.33 of "de Finibus Bonorum et Malorum" (The Extremes of Good and Evil) by Cicero, written in 45 BC. This book is a treatise on the theory of ethics, \textbf{very popular during the Renaissance. The first line of Lorem Ipsum, "Lorem ipsum dolor sit amet..", comes from a line in section 1.10.32.}
The standard chunk of Lorem Ipsum used since the 1500s is reproduced below for those interested. Sections 1.10.32 and 1.10.33 from "de Finibus Bonorum et Malorum" by Cicero are also reproduced in their exact original form, accompanied by English versions from the 1914 translation by H. Rackham.
	\subsection{Część główna}
	\label{Część główna}
	Czy znacie historię o niemym Michałku, który żuł gumę tak długo, aż oślepł?
	\part*{Rozdział 2}
	\section{Płaca w Chinach}
	 \(2 + 2 = 3+1\) 
\label{siema}
	\\Wielki Wybuch (ang. Big Bang) – najwcześniejsze znane wydarzenie w obserwowalnym Wszechświecie, jego najwcześniejsza znana faza (etap) ewolucji, a jednocześnie nazwa modelu tego procesu. Według tego scenariusza ok. 13,799 ± 0,021 mld lat temu[1] miał miejsce Wielki Wybuch – z bardzo gęstej i gorącej materii wyłonił się znany Wszechświat, tzn. obserwowana przez człowieka materia, energia i oddziaływania. W niektórych wariantach tego modelu Wielki Wybuch zaczyna się od początkowej osobliwości[2] i wyłania się z niego sama czasoprzestrzeń (przestrzeń i czas).

Teoria ta opiera się na obserwacjach wskazujących na rozszerzanie się przestrzeni zgodnie z metryką Friedmana-Lemaître’a-Robertsona-Walkera. Przemawia za tym przesunięcie ku czerwieni widma promieniowania elektromagnetycznego pochodzącego z odległych galaktyk, zgodne z prawem Hubble’a, w powiązaniu ze słabą zasadą kosmologiczną. Obserwacje te wskazują, że Wszechświat rozszerza się od stanu, w którym cała materia Wszechświata miała bardzo dużą gęstość i temperaturę, który jest identyfikowany z grawitacyjną osobliwością.
	\begin{table}[H]
		\centering
		\begin{tabular}{||c c ||} 
			\hline
			Ilość wybuchów & Moc \\ [1ex] 
			\hline\hline
			1 & Bardzo mocne \\ 
			\hline
			2 & Mocne \\ [1ex]
			\hline
		\end{tabular}
	\caption{Ile wybuchów}
	\label{table: ilewybuchów}
	\end{table}
	Sto lat, sto lat - niech żyje, żyje nam
Sto lat, sto lat - niech żyje, żyje nam!
Jeszcze raz, jeszcze raz - niech żyje, żyje nam
Niech żyje żyje nam
\begin{figure}[H]
		\centering
		\includegraphics[width=10cm, height=5cm]{opaska.jpg}
		\caption{Opaska}
		\label{fig: Opaska}
	\end{figure}
Sto lat, sto lat - sto lat, sto lat - niechaj żyje nam
Sto lat, sto lat - sto lat, sto lat - niechaj żyje nam
Niech żyje nam - niech żyje nam!
W zdrowiu szczęściu, pomyślności - niechaj żyje nam!
\[E=mc^2\]
Niech żyje nam, niech żyje nam
W zdrowiu szczęściu, pomyślności - niechaj żyje nam

	
	\subsection{O mieszkaniu na Litwie}
	\label{wlochy}
	Dlaczego warto pojechać do Włoch?  Powodów jest wiele, tymczasem wciąż słyszę od tych, którzy tam nie byli, takie opinie: Włochy są drogie, Włosi nie znają angielskiego, a na dodatek jeszcze taki jeden z drugim poderwie moją dziewczynę i wrócę z niczym. I jeszcze jest niebezpiecznie, okradną Cię, tyle się słyszy o kieszonkowcach, albo nawet gorzej – przyłożą Ci pistolet do głowy i oddawaj wszystko, co masz. Nie wspomnę już o mafii, przecież ona tam cały czas działa. Nie, nie, ja tam nie jadę. To kraj bufonów i bawidamków ciągle pijących wino. Gdyby tam jeszcze był nasz papież. Może jest  i ładnie, ale wolę jechać gdzie indziej.
	\begin{figure}[H]
		\centering
		\includegraphics[width=8cm, height=5cm]{polska.jpg}
		\caption{Polska}
		\label{fig: Polska}
	\end{figure}
	\label{pralka}
	Bawełniane T-shirty, koronkowe bluzki, śpiochy niemowlęce, strój na trening, kostiumy kąpielowe, wełniane swetry, a do tego dżinsy, dresy, pościel, ręczniki i ukochane pluszaki. Można by tak wymieniać bez końca. I choć pranie to jedna z najbardziej nielubianych domowych czynności, to nie da się jej uniknąć. \\underline{Wniosek nasuwa się sam: pralka jest niezbędna w każdym domu.} W naszym poradniku podpowiadamy, jak wybrać pralkę – na które parametry sprzętu zwrócić szczególną uwagę, by urządzenie jak najpełniej zaspokoiło wasze potrzeby.
	\newpage
	Dratewka był młodym i ubogim szewcem. Nie miał warsztatu, lecz wędrował po świecie i łatał stare obuwie. Miał jednak dobre serce. Pomagał każdemu stworzeniu spotkanemu na drodze. Kiedy zobaczył mrowisko, rozkopane przez niedźwiedzia, pomógł mrówkom je naprawić. Kiedy ujrzał zniszczoną pasiekę, także pomógł pszczołom doprowadzić ją do porządku. Z dzikimi kaczkami spotkanymi na stawie podzielił się chlebem. Wdzięczne zwierzęta obiecały mu pomoc, kiedy znajdzie się w potrzebie, jednak szewczyk zlekceważył ich obietnice, gdyż nie wierzył, że mogą one być pomocne w czymkolwiek.

Pewnego dnia doszedł do krainy, gdzie był wielki zamek. Okazało się, że mieszkała w nim zła czarownica. Dratewka dowiedział się od miejscowych, że więzi ona piękną dziewczynę. Wielu śmiałków próbowało ja uwolnić, ale wszyscy ginęli z rąk czarownicy. Dratewka postanowił spróbować pomóc pannie. Zapukał do drzwi zamku czarownicy i zażądał uwolnienia dziewczyny, oświadczając, że pragnie ją poślubić. Czarownica odpowiedziała, że uwolni pannę, jeśli szewczyk wykona dwie prace, które mu zleci i rozwiąże jedną zagadkę. Jeśli zaś nie podoła zadaniom, których się podejmie – straci głowę. Dratewka zaakceptował te warunki.

Najpierw czarownica zaprowadziła go do komnaty i pokazała szewczykowi worek piasku zmieszanego z makiem, kazała mu to rozdzielić do rana, inaczej groziła śmiercią. Dratewka posmutniał, gdyż nie był w stanie tego dokonać. Jednak wtedy przyszły do niego mrówki, którym kiedyś pomógł. Z ich pomocą praca została szybko wykonana.

Kolejnym zadaniem było znalezienie złotego kluczyka, zgubionego w stawie – Dratewka dokonał tego dzięki pomocy kaczek, z którymi kiedyś dzielił się chlebem.

Na koniec czarownica zaprowadziła go do komnaty, w której siedziało dziewięć panien. Wszystkie były jednakowo biało ubrane i miały zasłonięte głowy. Wiedźma kazała mu zgadnąć, która z nich to więziona panna. Dratewka nie miał pojęcia, ale z pomocą przyszli mu jego dawni przyjaciele – pszczoły, które wleciały przez okna i zaczęły krążyć wokół jednej z panien. Tę właśnie Dratewka wskazał. Wówczas czarownica zmieniła się w ptaka i odleciała, a uratowana panna rzuciła mu się na szyję, dziękując za ocalenie z niewoli. Oboje pobrali się i zamieszkali w zamku czarownicy.
\newpage
\label{krakow}
Wykopaliska archeologiczne dowodzą, że wzgórze wawelskie było zamieszkane już w epoce kamienia łupanego. Przypuszczalnie z VII w. pochodzą kopce Krakusa i Wandy legendarnych władców osady zamieszkałej przez słowiańskie plemiona Wiślan.
Pierwszą udokumentowaną wzmianką o Krakowie jest relacja kordobańskiego kupca Ibrahima ibn Jakuba z 965 r., w której wspomina o otoczonym lasami bogatym grodzie, leżącym na skrzyżowaniu szlaków handlowych. Z okresem przedpiastowskim związane są jeszcze dwie inne daty, istotne dla historii miasta: między rokiem 876 a 879 książę wielkomorawski Światopełk zajął późniejszą Małopolskę, a po roku 955 książę Bolesław Okrutny, brat św. Wacława, wprowadził rządy czeskie. W X w. Kraków włączono do tworzącego się państwa polskiego chociaż trudno jednoznacznie stwierdzić, czy stało się to za rządów Mieszka I w 990 r., czy Bolesława Chrobrego w 999 r.
Średniowiecze
We wczesnym średniowieczu na wzgórzu wawelskim funkcjonował umocniony drewnianym częstokołem i wałem obronnym gród z podgrodziem. W X i XI w. powstały pierwsze budowle murowane - zamek i romańskie kościoły: katedra i bazylika romańska oraz kościół św. Feliksa i Adaukta. W 1000 r. w Krakowie utworzono biskupstwo. Od 1150 r. przy kościele zamkowym działała szkoła katedralna - najlepsza polska uczelnia przed założeniem uniwersytetu. W skarbcu katedralnym przechowywano insygnia władzy: koronę i berło Bolesława Śmiałego. W bibliotece, która w 1100 r. liczyła 28 tytułów, przechowywano - oprócz ksiąg o tematyce religijnej - kilka dzieł literatury klasycznej, m.in. komedie Terencjusza, elegie Owidiusza i monografie historyczne Salustiusza
W 1142 r. biskup Robert konsekrował tzw. drugą katedrę wawelską, powstałą na miejscu zburzonego kościoła romańskiego. W reprezentacyjnej katedrze uroczyście złożono - po przeniesieniu ze Skałki - ciało św. Stanisława oraz relikwie św. Floriana.
Rozbicie dzielnicowe w XII w. i nieustanne walki książąt dzielnicowych nie przeszkodziły miastu w intensywnym rozwoju i rozbudowie. W 1138 r. Kraków nabrał większego znaczenia, zgodnie z testamentem Bolesława Krzywoustego został bowiem siedzibą senioratu i niejako stolicą Polski. Zniszczone po najeździe Tatarów w 1241 r. budowle zostały zastąpione przez nowe, w stylu gotyckim. Rok 1257 to kolejna ważna data w historii Krakowa - lokacji miasta na prawie magdeburskim, która wyznaczyła nowy układ urbanistyczny: z centralnie położonym rynkiem i regularną szachownicą rozchodzących się ulic. W XIII w. utworzono nowy system umocnień, z murami obronnymi, basztami strzelniczymi i ufortyfikowanymi bramami wjazdowymi, w następnych wiekach stopniowo wzbogacany i modyfikowany. 20 stycznia 1320 r. w Krakowie odbyła się pierwsza koronacja - Władysława Łokietka i jego żony Jadwigi. W następnych pięciu wiekach miało tu miejsce 35 innych koronacji. Katedra stała się również królewską nekropolią.
\newpage
 Odniesienie do tabeli z wybuchami ~\ref{table: ilewybuchów} 
\\Odniesienie do flagi Portugalii ~\ref{fig: Polska}
	\newpage
	BIBLIOGRAFIA\\
	 \url{https://pl.wikipedia.org/wiki/Wielki_Wybuch} ~\ref{siema}\\
	\url{https://www.italiapozaszlakiem.com/dlaczego-warto-pojechac-do-wloch/} ~\ref{wlochy}\\
	\url{https://www.euro.com.pl/artykuly/wszystkie/artykul-pralka-jak-wybrac.bhtml} ~\ref{pralka}\\
	\url{https://www.krakow.pl/nasze_miasto/1115,artykul,historia.html} ~\ref{krakow}
\end{document}