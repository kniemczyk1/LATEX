
\documentclass{beamer}
\usepackage[polish]{babel}
\usepackage[utf8]{inputenc}
\usepackage[T1]{fontenc}
\usepackage{wrapfig}
\graphicspath{ {./images/} }
\usetheme{Hannover}
\usecolortheme{wolverine}
\title{KAMIENIE}
\author{Kamil Niemczyk}
\date{30.01.2022}

\begin{document}
\begin{frame}
\titlepage
\end{frame}
\begin{frame}{Kamienie co to jest}
Kamienie są tą minerały\cite{Kamienie}
\pause
\begin{center}
\includegraphics[scale=0.30]{kamienie.jpg}
\label{fig: kamienie}
\end{center}
\end{frame}
\begin{frame}{Jakie są kamienie}
\begin{enumerate}
\item Twarde
\pause
\item Mocne
\pause
\item Trudno je zniszczyć
\pause
\item Potrafią być ciężkie
\pause
\item Można nimi rzucać, ale jest to niebezpieczne
\end{enumerate}
\end{frame}
\begin{frame}{Diament}
\begin{center}
\includegraphics[scale=0.2]{diament.jpg}\\
\label{fig: diament}
\end{center}
Diamenty można sprzedać za dużo pieniędzy
\end{frame}
\begin{frame}{Węgiel}
\begin{center}
\includegraphics[scale=0.1]{wegiel.jpg}
\label{fig: węgiel}
\end{center}
\end{frame}
\begin{frame}{Kamienie na szaniec}
Niby w nazwie kamienie, a to książka
\pause
\begin{center}
\includegraphics[scale=0.25]{naszaniec.jpg}
\label{fig: naszaniec}
\end{center}
\end{frame}
\begin{frame}{Ciekawostki o Kamieniach na szaniec}
\begin{enumerate}
\item Aleksander Kamiński jest autorem
\pause
\item Pierwsze wydanie w 1943 roku
\pause
\item Napisane w języku polskim
\end{enumerate}
\end{frame}
\begin{frame}{Kamien milowy}
Kamień milowy to taki znacznik na drodze\cite{Kamienie}
\pause
\begin{center}
\includegraphics[scale=0.2]{milowy.jpg}
\label{fig: milowy}
\end{center}
\end{frame}
\begin{frame}{Kamień murowy}
Kamień murowy, a dokładnie Kamień murowy granitowy łupany jasnoszary STRZEGOM\cite{Kamienie}
\pause
\begin{center}
\includegraphics[scale=0.50]{murowy.jpg}
\label{fig: murowy}
\end{center}
\end{frame}
\begin{frame}{Kamień diabelski}
Diabelski kamień w olszynie\cite{Kamienie}
\pause
\begin{center}
\includegraphics[scale=0.50]{diabel.jpg}
\label{fig: diabelski}
\end{center}
\end{frame}
\begin{frame}{Kamień z wyspy wielkanocnej}
A tam tak przeciągają kamienie\cite{Kamienie}
\pause
\begin{center}
\includegraphics[scale=0.2]{wyspa.jpg}
\label{fig: wielkanocny}
\end{center}
\end{frame} 
\begin{frame}{Tabelka dotycząca kamieni}
\begin{table}[H]
\label{table: tab}
		\begin{tabular}{||c c ||} 
			\hline
			Jaki Kamień & Ocena od 0 do 10 \\ [1ex] 
			\hline\hline
			Diament & 10/10 \\ 
			\hline
			Wielkanocny & 1/10 \\ 
			\hline
			Milowy & 0/10 \\ 
			\hline
			Na szaniec & 10/10 \\ 
			\hline
			STRZEGOM & 14/10 \\ 
			\hline
			Diabelski & 7/10 \\ 
			\hline
			Węgiel & 5/10 \\ [1ex]
			\hline
		\end{tabular}
	\end{table}
\end{frame}
\begin{frame}{Zestawienie kamieni top3}
\begin{enumerate}
\item SZTRZEGOM
\pause
\item Diament i Na szaniec
\pause
\item Diabelski
\end{enumerate}
\end{frame}
\begin{thebibliography}{4}
\bibitem{Kamienie}
\url{https://www.google.com/search?q=kamienie&sxsrf=APq-WBuir33C2AmciwFT7X8771Bb3I6kiA:1643572099875&source=lnms&tbm=isch&sa=X&ved=2ahUKEwi1srOZn9r1AhVjlosKHQbbAWYQ_AUoAXoECAIQAw&biw=1920&bih=969&dpr=1}
\end{thebibliography}
A TUTAJ ODWOŁANIA:\\
Odniesienie do kamieni ~\ref{fig: kamienie}\\
Odniesienie do Kamieni na szaniec ~\ref{fig: naszaniec}\\
Odniesienie do kamienia milowego ~\ref{fig: milowy}\\
Odniesienie do kamienia STRZEGOM ~\ref{fig: murowy}\\
Odniesienie do diabelskiego ~\ref{fig: diabelski}\\
Odniesienie do wielkanocnego ~\ref{fig: wielkanocny}\\
Odwołanie do tabeli ~\ref{table: tab}
\end{document}